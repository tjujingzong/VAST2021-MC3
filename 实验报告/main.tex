
\documentclass[a4paper]{article}

\input{style/ch_xelatex.tex}


\lstset{frame=, basicstyle={\footnotesize\ttfamily}}



\graphicspath{ {images/} }
\usepackage{ctex}



\usepackage[square,numbers]{natbib}
\bibliographystyle{abbrvnat}


\begin{document}
\renewcommand{\contentsname}{目\ 录}
\renewcommand{\appendixname}{附录}
\renewcommand{\appendixpagename}{附录}
\renewcommand{\refname}{参考文献}

\renewcommand{\tablename}{表}
\renewcommand{\today}{\number\year 年 \number\month 月 \number\day 日}

\title{{\Huge 可视语言与信息可视化{\large\linebreak\\}}{\Large 团队ID: \linebreak}
{\Large  \linebreak\linebreak}}
\author{ \\
\\\\\\
天津大学,智能与计算学部}
\date{\today}
\maketitle
\newpage

\begin{center}
\tableofcontents\label{c}
\end{center}
\newpage


\begin{center}
{\Large\bf{摘\ 要\\}}

大作业要求大家按照论文短文的格式进行书写,参考文献~\cite{bayrak2020pragma, govyadinov2019graph}。



\end{center}

\newpage



\section{引言}
\label{overview}

描述问题、技术挑战

\section{相关工作}

调研相关论文发表,搜索CNKI或者Google scholar 等学术引擎,了解该领域研究现状。参考文献格式为~\cite{bayrak2020pragma}。

\section{问题描述和需求分析}
\label{Data and Task Abstraction}


\section{解决方案}


\section{实验结果和案例分析}\label{sub:ptxeva}


\section{总结}


\bibliography{ref}

\end{document}

